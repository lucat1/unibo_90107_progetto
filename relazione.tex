\documentclass[a4paper,11pt]{article}
\usepackage[utf8]{inputenc}
\usepackage[italian]{babel}
\usepackage{tabularx}
\usepackage{graphicx}

\title{Progetto Database: organizzatore eventi}
\author{Luca Bassi, Gabriele Genovese, Gianmaria Rovelli, Luca Tagliavini}
\date{Dicembre 2022}

\begin{document}

\maketitle

\tableofcontents

\section{Analisi dei requisiti}

\subsection{Requisiti espressi in linguaggio naturale}

Vogliamo realizzare un database per un'azienda di organizzazione di eventi e spettacoli, che comprende la gestione dei biglietti, dei luoghi e del personale necessario alla realizzazione.
Un artista è rappresentato come un nome. Uno spettacolo è rappresentato da un titolo, un artista, dal costo della SIAE e da un chacet. Uno spettacolo è una serie di eventi e di ogni evento vogliamo memorizzare il titolo, il luogo, la data di inizio e fine, oltre allo spettacolo al quale è associato. L'azienda può organizzare gli eventi in infrastrutture diverse (palazzetti, stadi, fiere, parchi pubblici, etc...) di conseguenza serve memorizzare un luogo di cui specificare un nome, un tipo (rappresentato da un nome), le coordinate e un prezzo giornaliero per l'affitto. Un luogo potrebbe avere posti (a sedere o in piedi) più convenienti o confortevoli per vendere biglietti a prezzo più alto, quindi rappresentiamo un settore di un luogo con la sua capienza, un nome e collegandolo a un luogo. Un posto di un settore viene identificato da una codice (che è una coppia riga e colonna). Il prezzo dei biglietti associati a un settore di un luogo può variare a seconda dell'evento. Ogni biglietto è quindi rappresentato da un posto a sedere e un evento. Infine, vogliamo tenere traccia di tutti i servizi erogati dall'azienda (maschere, facchini, biglietteria, sicurezza) e richiesti dagli artisti (tecnico delle luci e del suono). Il tipo di un servizio è rappresentato da un nome e da una descrizione. Il fornitore di un servizio è rappresentato dal tipo e da un prezzo all'ora.

\subsection{Glossario dei termini}

\begin{tabularx}{\textwidth}{|X|>{\raggedright\arraybackslash}X|X|>{\raggedright\arraybackslash}X|}
\hline
\textbf{Termine} & \textbf{Descrizione} & \textbf{Sinonimi} & \textbf{Collegamenti}\\
\hline
Evento & La singola data di uno spettacolo o di un concerto & Event & Spettacolo, Luogo, Settore, Biglietto, Servizio\\
\hline
Spettacolo & L'insieme di eventi di un determinato artista & Concerto, Show & Evento, Artista\\
\hline
Artista & Singola persona o gruppo di persone che esegue uno spettacolo & Artist & Spettacolo\\
\hline
Luogo & Edificio, piazza o parco (privato/pubblico) nel quale si organizza uno spettacolo & Location, Venue & Evento, Settore\\
\hline
Settore & Le parti in cui sono suddivisi i posti in un luogo & Sector & Evento, Luogo, Posto\\
\hline
Posto & Posto (a sedere o in piedi) in un settore & Seat & Settore, Biglietto\\
\hline
Biglietto & Il biglietto di ingresso per un evento & Ticket & Evento, Posto\\
\hline
Servizio & Un servizio richiesto per un evento & Service & Evento, Fornitore\\
\hline
Fornitore & Un fornitore di servizi & Provider & Servizio\\
\hline
\end{tabularx}

\subsection{Eliminazione delle ambiguità presenti}
TODO: capire se necessario


\subsection{Strutturazione dei requisiti}

\subsubsection*{Frasi di carattere generale}

Vogliamo realizzare un database per un'azienda di organizzazione di eventi e spettacoli, che comprende la gestione dei biglietti, dei luoghi e del personale necessario alla realizzazione.

\subsubsection*{Frasi relative agli artisti}

Un artista è rappresentato come un nome.

\subsubsection*{Frasi relative agli spettacoli e agli eventi}

Uno spettacolo è una serie di eventi e di ogni evento vogliamo memorizzare il titolo, il luogo, la data di inizio e fine, oltre allo spettacolo al quale è associato.

\subsubsection*{Frasi relative ai luoghi}

L'azienda può organizzare gli eventi in infrastrutture diverse (palazzetti, stadi, fiere, parchi pubblici, etc...) di conseguenza serve memorizzare un luogo di cui specificare un nome, un tipo (rappresentato da un nome), le coordinate e un prezzo giornaliero per l'affitto.

\subsubsection*{Frasi relative ai settori}

Un luogo potrebbe avere posti (a sedere o in piedi) più convenienti o confortevoli per vendere biglietti a prezzo più alto, quindi rappresentiamo un settore di un luogo con la sua capienza, un nome e collegandolo a un luogo.

\subsubsection*{Frasi relative ai posti}

Un posto di un settore viene identificato da una codice (che è una coppia riga e colonna).

\subsubsection*{Frasi relative ai biglietti}

Il prezzo dei biglietti associati a un settore di un luogo può variare a seconda dell'evento. Ogni biglietto è quindi rappresentato da un posto a sedere e un evento.

\subsubsection*{Frasi relative ai servizi}

Infine, vogliamo tenere traccia di tutti i servizi erogati dall'azienda (maschere, facchini, biglietteria, sicurezza) e richiesti dagli artisti (tecnico delle luci e del suono). Il tipo di un servizio è rappresentato da un nome e da una descrizione.

\subsubsection*{Frasi relative ai fornitori}

Il fornitore di un servizio è rappresentato dal tipo e da un prezzo all'ora.

\subsection{Specifica operazioni}

\begin{enumerate}
    \item Inserire un artista (in media 10 volte all'anno)
    \item Inserire uno spettacolo (in media 35 volte all'anno)
    \item Inserire un fornitore di servizi (in media 5 volte all'anno)
    \item Inserire un luogo suddiviso in settori (in media 3 volte all'anno)
    \item Inserire un evento (in media 50 volte all'anno)
    \item Prenotare un servizio (in media 200 volte all'anno)
    \item Vendere un biglietto (in media 30 000 volte all'anno)
    \item Rimborsare un biglietto (in media 300 volte l'anno)
    \item Trovare un biglietto dato il posto (in media 28 000 volte all'anno)
    \item Visualizzare tutti gli eventi di un artista (in media 50 000 all'anno)
    \item Visualizzare tutti gli eventi di uno spettacolo (in media 10 000 all'anno)
    \item Visualizzare le spese di un evento (in media 50 volte all'anno)
    \item Visualizzare gli incassi di un evento (in media 50 volte all'anno)
    \item Visualizzare i guadagni di un evento (in media 50 volte all'anno)
    \item Visualizzare il numero di posti liberi di un determinato evento (in media 1000 volte l'anno)
    \item Visualizzare la percentuale media di biglietti venduti da un determinato artista per ogni evento (in media 100 volte l'anno)
    \item Visualizzare quali eventi hanno fatto sold out (in media 50 volte l'anno)
    \item Visualizzare gli eventi presenti in un arco temporale
    \item Visualizzare tutti gli eventi che si svolgeranno in un determinato luogo
    \item Visualizzare lo spettacolo che ha guadagnato di più nell'ultimo periodo
\end{enumerate}

\section{Progettazione concettuale}

\subsection{Identificazione delle entità e relazioni}

Abbiamo identificato le seguenti entità: evento, spettacolo, artista, luogo, settore, posto biglietto, servizio, fornitore.

\subsection{Un primo scheletro dello schema}

A un livello di astrazione più alto abbiamo i biglietti venduti per un evento.

\includegraphics[width=\textwidth]{base.png}
 
\subsection{Sviluppo delle componenti dello scheletro}

\subsubsection*{Spettacolo e artista}

Abbiamo poi sviluppato lo spettacolo come un'entità con un titolo, un cachet, un costo SIAE ed è composto da vari eventi, ognuno con un titolo, un luogo, un inizio e una fine.\\
Un evento è composto da un solo spettacolo.

Ogni spettacolo è eseguito da un artista che ha un nome, un artista può eseguire più spettacoli.

\includegraphics[width=\textwidth]{EventShowArtist.png}

\subsubsection*{Fornitore}

Ogni fornitore può offrire servizi a più eventi e ogni evento può avere più fornitori di servizi.

\includegraphics[width=\textwidth]{EventShowArtistProvider.png}

\subsection{Unione delle componenti nello schema finale ridotto}

\subsection{Dizionario dei dati}

\subsection{Regole aziendali}

\section{Progettazione logica}

\subsection{Tavole dei volumi e delle operazioni}

\subsection{Ristrutturazione dello schema concettuale}

\subsection{Normalizzazione}

\subsection{Traduzione verso il modello relazionale}

\section{Codice SQL}

\subsection{Definizione dello schema}

\subsection{Codifica delle operazioni}

\section{Esempi di interrogazioni}

\end{document}
