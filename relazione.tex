\documentclass[a4paper,11pt]{article}
\usepackage{fontspec}
\usepackage[english,italian]{babel}
\setcounter{tocdepth}{2}
\usepackage{tabularx}
\usepackage[hidelinks,colorlinks=true,linkcolor=black,urlcolor=blue]{hyperref}
\usepackage{graphicx}
\usepackage{enumitem}
\usepackage{csvsimple}

\usepackage{listings}
\setmonofont{IBM Plex Mono}
\lstset{
  basicstyle=\footnotesize\ttfamily,
  breakatwhitespace=false,
  breaklines=true,
  % commentstyle=\color{dkgreen},
  deletekeywords={...},
  escapeinside={\%*}{*)},
  extendedchars=true,
  % frame=single,
  keepspaces=true,
  language=SQL,
  otherkeywords={EXISTS},
  morekeywords={*,MODIFY,text,serial,...},
  % keywordstyle=\keywordcheck,
  % identifierstyle=\setidcolor,
  % ew
  tabsize=4
}

\title{\textbf{Progetto Basi di Dati \\ \large Organizzazione d'Eventi}}
\author{
  \begin{tabular}{l l l}
    Luca Bassi & \#916386 & \href{mailto:luca.bassi14@studio.unibo.it}{luca.bassi14@studio.unibo.it}\\
    Gabriele Genovese & \#970718 & \href{mailto:gabriele.genovese2@studio.unibo.it}{gabriele.genovese2@studio.unibo.it}\\
    Gianmaria Rovelli & \#970936 & \href{mailto:gianmaria.rovelli2@studio.unibo.it}{gianmaria.rovelli2@studio.unibo.it}\\
    Luca Tagliavini & \#971133 & \href{mailto:luca.tagliavini5@studio.unibo.it}{luca.tagliavini5@studio.unibo.it}
  \end{tabular}
}
\date{Dicembre 2022}

\begin{document}

\maketitle

\tableofcontents

\newpage

\section{Analisi dei requisiti}

\subsection{Requisiti espressi in linguaggio naturale}

Vogliamo realizzare un database per un'azienda di organizzazione di eventi e spettacoli, che comprende la gestione dei biglietti, dei luoghi e del personale necessario alla realizzazione.\\
Un artista ha un nome d'arte.\\
Uno spettacolo è rappresentato da un titolo, un artista, il prezzo della SIAE e da un cachet. Uno spettacolo è una serie di eventi e di ogni evento vogliamo memorizzare il titolo, il luogo, la data di inizio e fine, oltre allo spettacolo al quale è associato.\\
L'azienda può organizzare gli eventi in infrastrutture diverse (palazzetti, stadi, piazze, teatri, etc.) di conseguenza serve memorizzare un luogo di cui specificare un nome, un tipo (rappresentato da un nome), l'indirizzo, la città e un prezzo giornaliero per l'affitto.\\
Un luogo potrebbe avere posti (a sedere o in piedi) più convenienti o confortevoli per vendere biglietti a prezzo più alto, quindi rappresentiamo un settore di un luogo con la sua capienza, un nome e collegandolo a un luogo.\\
Un posto di un settore viene identificato da un codice (che è una coppia fila-numero).\\
Il prezzo dei biglietti associati a un settore di un luogo può variare a seconda dell'evento. Ogni biglietto ha un nominativo ed è associato a un posto e a un evento.\\
Infine, vogliamo tenere traccia di tutti i servizi erogati dalle aziende (maschere, facchini, biglietteria, sicurezza) e richiesti dagli artisti (tecnico delle luci e del suono). I fornitori offrono servizi che hanno un tipo e un prezzo.

\subsection{Glossario dei termini}

\begin{tabularx}{\textwidth}{|X|>{\raggedright\arraybackslash}X|X|>{\raggedright\arraybackslash}X|}
\hline
\textbf{Termine} & \textbf{Descrizione} & \textbf{Sinonimi} & \textbf{Collegamenti}\\
\hline
Evento & La singola data di uno spettacolo o di un concerto &  & Spettacolo, Luogo, Settore, Biglietto, Fornitore\\
\hline
Spettacolo & L'insieme di eventi di un determinato artista & Concerto & Evento, Artista\\
\hline
Artista & Singola persona o gruppo di persone che esegue uno spettacolo & Gruppo, Compagnia teatrale & Spettacolo\\
\hline
Luogo & Edificio, piazza o parco (privato/pubblico) nel quale si organizza uno spettacolo & Spazio & Evento, Settore\\
\hline
Settore & Le parti in cui sono suddivisi i posti in un luogo & Divisioni & Evento, Luogo, Posto\\
\hline
Posto & Posto (a sedere o in piedi) in un settore & Sedia & Settore, Biglietto\\
\hline
Biglietto & Il biglietto di ingresso per un evento &  & Evento, Posto\\
\hline
Fornitore & Un fornitore di servizi & Azienda & Evento\\
\hline
\end{tabularx}

\subsection{Eliminazione delle ambiguità presenti}

Dall'analisi del termine artista abbiamo convenuto che può essere una persona o un gruppo a cui appartengo più persone.\\
Le persone hanno un nome, un cognome e una data di nascita e possono appartenere a uno o più gruppi.\\
I gruppi hanno una data di formazione.

\subsection{Strutturazione dei requisiti}

\subsubsection*{Frasi di carattere generale}

Vogliamo realizzare un database per un'azienda di organizzazione di eventi e spettacoli, che comprende la gestione dei biglietti, dei luoghi e del personale necessario alla realizzazione.

\subsubsection*{Frasi relative agli artisti}

Un artista ha un nome d'arte.\\
Un'artista che può essere una persona o un gruppo a cui appartengo più persone.
Le persone hanno un nome, un cognome e una data di nascita e possono appartenere a uno o più gruppi.
I gruppi hanno una data di formazione.

\subsubsection*{Frasi relative agli spettacoli e agli eventi}

Uno spettacolo è rappresentato da un titolo, un artista, il prezzo della SIAE e da un cachet. Uno spettacolo è una serie di eventi e di ogni evento vogliamo memorizzare il titolo, il luogo, la data di inizio e fine, oltre allo spettacolo al quale è associato.

\subsubsection*{Frasi relative ai luoghi}

L'azienda può organizzare gli eventi in infrastrutture diverse (palazzetti, stadi, piazze, teatri, etc.) di conseguenza serve memorizzare un luogo di cui specificare un nome, un tipo (rappresentato da un nome), l'indirizzo, la città e un prezzo giornaliero per l'affitto.

\subsubsection*{Frasi relative ai settori}

Un luogo potrebbe avere posti (a sedere o in piedi) più convenienti o confortevoli per vendere biglietti a prezzo più alto, quindi rappresentiamo un settore di un luogo con la sua capienza, un nome e collegandolo a un luogo.

\subsubsection*{Frasi relative ai posti}

Un posto di un settore viene identificato da un codice (che è una coppia fila-numero).

\subsubsection*{Frasi relative ai biglietti}

Il prezzo dei biglietti associati a un settore di un luogo può variare a seconda dell'evento. Ogni biglietto ha un nominativo ed è associato a un posto e a un evento.

\subsubsection*{Frasi relative ai fornitori}

Infine, vogliamo tenere traccia di tutti i servizi erogati dalle aziende (maschere, facchini, biglietteria, sicurezza) e richiesti dagli artisti (tecnico delle luci e del suono). I fornitori offrono servizi che hanno un tipo e un prezzo.

\subsection{Specifica operazioni}

\begin{enumerate}
    \item Inserire un artista (in media 10 volte all'anno)
    \item Inserire uno spettacolo (in media 10 volte all'anno)
    \item Inserire un fornitore di servizi (in media 5 volte all'anno)
    \item Inserire un luogo suddiviso in settori (in media 3 volte all'anno)
    \item \label{oploc1} Inserire un evento (in media 50 volte all'anno)
    \item \label{opsec1} Assegnare un prezzo a ogni settore per un evento (in media 300 volte all'anno)
    \item Prenotare un servizio per un evento (in media 200 volte all'anno)
    \item \label{oploc2} Visualizzare un evento con il luogo dove si svolgerà (in media 25 000 volte all'anno)
    \item Vendere un biglietto (in media 17 500 volte all'anno)
    \item Rimborsare un biglietto (in media 20 volte al mese)
    \item Trovare un biglietto dato il posto e l'evento (in media 2 300 volte al mese)
    \item Visualizzare la capienza totale di un luogo (in media 6 volte al mese)
    \item \label{search-venue-capacity-limit} Visualizzare i luoghi che soddisfano una capienza (in media 3 volte al mese)
    \item \label{oploc3} Visualizzare tutti gli eventi di un artista (in media 15 volte al giorno)
    \item Visualizzare tutti gli eventi di uno spettacolo (in media 1 000 volte al mese)
    \item \label{oploc4} Visualizzare le spese di un evento (in media 8 volte al mese)
    \item Visualizzare gli incassi di un evento (in media 2 volte al mese)
    \item \label{oploc5} Visualizzare i guadagni di un evento (in media 5 volte al mese)
    \item Visualizzare informazioni sui posti di un determinato evento (in media 1 750 volte al mese)
    \item Visualizzare la percentuale media di biglietti venduti da un determinato artista (in media 100 volte all'anno)
    \item Visualizzare quali eventi hanno fatto sold out (in media 20 volte all'anno)
    \item Visualizzare gli eventi presenti in un arco temporale (in media 50 volte al giorno)
    \item Visualizzare tutti gli eventi che si svolgeranno in un determinato luogo (in media 10 volte al giorno)
    \item Visualizzare i guadagni dato uno spettacolo (in media 50 volte all'anno)
    \item Visualizzare lo spettacolo che ha guadagnato di più (in media 20 volte all'anno)
\end{enumerate}

\section{Progettazione concettuale}

\subsection{Identificazione delle entità e relazioni (bottom-up)}

Abbiamo identificato le seguenti entità: evento, spettacolo, artista (persona e gruppo), fornitore, luogo, settore, posto e biglietto.

Possiamo dividere le entità in quattro gruppi:

\begin{itemize}
    \item Spettacolo e artista (persona e gruppo).
    \item Fornitore.
    \item Luogo, settore, posto.
    \item Biglietto.
\end{itemize}

\pagebreak
\subsection{Un primo scheletro dello schema (top-down)}

A un livello di astrazione più alto abbiamo un evento, parte di uno spettacolo di un artista, che si svolge in un luogo, di cui vengono venduti i biglietti e in cui ci sono dei fornitori che offrono dei servizi.

\vspace*{1em}
\includegraphics[width=\textwidth]{ERScheletro.png}
 
\subsection{Sviluppo delle componenti dello scheletro}

\subsubsection*{Spettacolo e artista}

Abbiamo poi sviluppato lo spettacolo come un'entità con un titolo, un cachet, il prezzo della SIAE ed è composto da vari eventi, ognuno con un titolo, un inizio e una fine.\\
Un evento è composto da un solo spettacolo.\\
Ogni spettacolo è eseguito da un artista che ha un nome d'arte, un artista può eseguire più spettacoli.
Un artista può essere una persona (con nome, cognome e data di nascita) o un gruppo (con una data di formazione) a cui appartengono una o più persone.

\includegraphics[width=\textwidth]{ERSpettacoloArtista.png}

\subsubsection*{Fornitore}

Ogni fornitore può offrire servizi a più eventi e ogni evento può avere più fornitori di servizi, i servizi sono di diverse tipologie.\\
Ogni fornitore può fornire vari tipi di servizi e ogni evento può aver necessità di diversi tipi di servizi.

\includegraphics[width=\textwidth]{ERFornitore.png}

\subsubsection*{Luogo, settore, posto}

Ogni luogo ha un nome, un indirizzo, la città, un tipo e un prezzo per l'affitto.
Ogni evento si svolge in un luogo e un luogo può ospitare più eventi.\\
Ogni luogo è diviso in settori.
Ogni settore ha un nome (che è univoco all'interno del luogo), una capienza massima e un prezzo dei biglietti per ogni evento.\\
In ogni settore sono presenti dei posti segnati con la lettera della fila e un numero (univoci all'interno del loro settore).

\vspace*{.5em}
\includegraphics[width=\textwidth]{ERLuogo.png}

\subsubsection*{Biglietto}

Di ogni evento vendiamo diversi biglietti, ma ogni biglietto è valido per un solo evento.
Ogni biglietto ha un codice univoco e un nominativo.

\vspace*{.5em}
\includegraphics[width=\textwidth]{ERBiglietto.png}

\subsection{Unione delle componenti nello schema finale}

Ogni biglietto è associato a un posto.

\vspace*{1em}
\includegraphics[width=\textwidth]{ER.png}

\subsection{Dizionario dei dati}

\subsubsection*{Entità}

\begin{tabularx}{\textwidth}{|>{\hsize=.6\hsize}X|>{\raggedright\arraybackslash}X|>{\raggedright\arraybackslash\hsize=1.4\hsize}X|>{\raggedright\arraybackslash}X|}
\hline
\textbf{Nome} & \textbf{Descrizione} & \textbf{Attributi} & \textbf{Identificatori}\\
\hline
Persona & & Nome (stringa), cognome (stringa), data di nascita (data) &\\
\hline
Gruppo & A cui appartengono delle persone & Data di formazione (data) &\\
\hline
Artista & Un artista (persona o gruppo) che esegue degli spettacoli & \textit{Persona} (riferimento), \textit{gruppo} (riferimento) & Nome d'arte (stringa)\\
\hline
Spettacolo & Un serie di eventi di un artista & \textit{Artista} (riferimento), cachet (float), prezzo SIAE (float) & Titolo (stringa)\\
\hline
Luogo & Dove si svolge un evento & Prezzo (float), tipo (enum) & Nome (stringa), indirizzo (stringa), città (stringa)\\
\hline
Evento & Una singola data di uno spettacolo & Data e ora di fine (timestamp), \textit{spettacolo} (riferimento) & Titolo (stringa), Data e ora di inizio (timestamp) e \textit{luogo} (riferimento)\\
\hline
Settore & Un settore di un luogo & Capienza (numero) & Nome (stringa) e \textit{luogo} (riferimento)\\
\hline
Posto & Un posto in un settore & & Fila (carattere), numero (numero) e \textit{settore} (riferimento)\\
\hline
Fornitore & Un fornitore di servizi & Descrizione (stringa) & Nome (stringa)\\
\hline
Biglietto & Il biglietto di ingresso per un evento & Nominativo (stringa), \textit{posto} (riferimento), \textit{evento} (riferimento) & Codice (numero)\\
\hline
\end{tabularx}

\subsubsection*{Relazioni}

\begin{tabularx}{\textwidth}{|X|>{\raggedright\arraybackslash}X|>{\raggedright\arraybackslash}X|>{\raggedright\arraybackslash}X|}
\hline
\textbf{Nome} & \textbf{Descrizione} & \textbf{Attributi} & \textbf{Entità coinvolte}\\
\hline
Appartenenza & Associa una persona a un gruppo & & Persona (0,N), Gruppo (1,N)\\
\hline
Esibizione & Associa un artista a uno spettacolo & & Artista (0,N), Spettacolo (1,1)\\
\hline
Data & Associa un evento con uno spettacolo & & Evento (1,1), Spettacolo (1,N)\\
\hline
Posizione & Associa un evento a un luogo & & Evento (1,1), Luogo (1,N)\\
\hline
Divisione & Associa un settore a un luogo & & Luogo (1,N), Settore (1,1)\\
\hline
Piantina & Associa un posto a un settore & & Settore(1,N), Posto (1,1)\\
\hline
Costo & Associa un settore a un evento & Prezzo (float) & Evento (1,N), Settore (1,N)\\
\hline
Vendita & Associa un biglietto a un evento & & Evento (1,N), Biglietto (1,1)\\
\hline
Prenotazione & Associa un biglietto a un posto & & Biglietto (1,1), Posto (1,N)\\
\hline
Servizio & Associa un fornitore a un evento & Tipo (enum), prezzo (float) & Fornitore (0,N), Evento (1,N)\\
\hline
\end{tabularx}

\subsection{Regole aziendali}

\textbf{Vincoli}

\begin{enumerate}[label=$\bullet$\hspace{.5em}(RV\arabic*),leftmargin=6em,ref=RV\theenumi]
  \item Il cachet e il prezzo SIAE di uno spettacolo devono essere maggiori o uguali a 0.
  \item Il prezzo di affitto di un luogo deve essere maggiore o uguale a 0.
  \item La fine di un evento dev'essere successiva all'inizio.
  \item La capienza di un settore deve essere maggiore di 0.
  \item Il prezzo di un settore deve essere maggiore o uguale a 0.
  \item Il prezzo di un servizio deve essere maggiore di 0.
  \item Il numero di posti in un settore è pari alla capienza.
  \item \label{rv7} Ogni settore del luogo di un evento deve avere un prezzo assegnato.
  \item \label{rv8} Tutti i settori collegati allo stesso evento devono appartenere allo stesso luogo.
\end{enumerate}

\textbf{Derivazioni}

\begin{enumerate}[label=$\bullet$\hspace{.5em}(RD\arabic*),leftmargin=6em,ref=RV\theenumi]
  \item Il prezzo di un biglietto è il costo del settore in cui si trova il posto del biglietto.
  \item Il costo di un evento è la somma tra il cachet e il prezzo della SIAE per l'affitto, il prezzo di affitto del luogo e i prezzi dei servizi.
  \item Il guadagno di un evento è la somma dei prezzi dei biglietti venduti meno il costo dell'evento.
\end{enumerate}

\section{Progettazione logica}

\subsection{Tavole dei volumi e delle operazioni}

\subsubsection*{Tavole dei volumi}

\begin{tabularx}{\textwidth}{|>{\hsize=.6\hsize}X|>{\hsize=.4\hsize}X|>{\raggedright\arraybackslash\hsize=2\hsize}X|}
\hline
\textbf{Concetto} & \textbf{Tipo} & \textbf{Volume}\\
\hline
Persona & E & 100\\
\hline
Gruppo & E & 100\\
\hline
Artista & E & 200\\
\hline
Spettacolo & E & 600\\
\hline
Luogo & E & 100\\
\hline
Evento & E & 1 000\\
\hline
Settore & E & 600\\
\hline
Posto & E & 50 000\\
\hline
Fornitore & E & 100\\
\hline
Biglietto & E & 350 000 ($\leq$ eventi · posti / luoghi)\\
\hline
Esibizione & R & 600\\
\hline
Data & R & 1 000 (= eventi)\\
\hline
Servizio & R & 5 000\\
\hline
Posizione & R & 1 000 (= eventi)\\
\hline
Divisione & R & 600 (= settori)\\
\hline
Piantina & R & 50 000 (= posti)\\
\hline
Costo & R & 6 000 (= eventi · settori / luoghi)\\
\hline
Vendita & R & 350 000 (= biglietto)\\
\hline
Prenotazione & R & 350 000 (= biglietto)\\
\hline
\end{tabularx}

\subsubsection*{Tavola delle operazioni}

\begin{tabularx}{\textwidth}{|X|>{\raggedright\arraybackslash}X|}
  \hline
  \textbf{Operazione} & \textbf{Frequenza}\\
  \hline
  1 & 10 volte all'anno\\
  \hline
  2 & 10 volte all'anno\\
  \hline
  3 & 5 volte all'anno\\
  \hline
  4 & 3 volte all'anno\\
  \hline
  5 & 50 volte all'anno\\
  \hline
  6 & 300 volte all'anno\\
  \hline
  7 & 200 volte all'anno\\
  \hline
  8 & 25 000 volte all'anno\\
  \hline
  9 & 17 500 volte all'anno\\
  \hline
  10 & 20 volte al mese\\
  \hline
  11 & 2 300 volte al mese\\
  \hline
  12 & 6 volte al mese\\
  \hline
  13 & 3 volte al mese \\
  \hline
  14 & 15 volte al giorno\\
  \hline
  15 & 1 000 volte al mese\\
  \hline
  16 & 8 volte al mese\\
  \hline
  17 & 2 volte al mese\\
  \hline
  18 & 5 volte al mese\\
  \hline
  19 & 1 750 volte al mese\\
  \hline
  20 & 100 volte all'anno\\
  \hline
  21 & 20 volte all'anno\\
  \hline
  22 & 50 volte al giorno\\
  \hline
  23 & 10 volte al giorno\\
  \hline
  24 & 50 volte all'anno\\
  \hline
  25 & 20 volte all'anno\\
  \hline
\end{tabularx}

\subsection{Ristrutturazione dello schema concettuale}

\subsubsection*{Studio delle ridondanze}

Dal diagramma ER traspare una evidente ridondanza nella relazione \emph{Posizione}, che lega un Evento a un Luogo.
Per trovare il luogo in cui si tiene un dato evento si può anche sfruttare la relazione \emph{Costo} e la relazione \emph{Divisione}.
Poiché quando si visualizza un evento si vuole molto spesso conoscere anche il luogo in cui si svolge, l'utilizzo di questa operazione è molto alto.
Per migliorare le prestazioni abbiamo preferito dunque lasciare una relazione ridondante, imponendo gli opportuni vincoli affinché non si verifichino inconsistenze (vedi \ref{rv7}, \ref{rv8}).

\subsubsection*{Tavole degli accessi mantenendo la ridondanza}

\begin{quote}
  Nota: rimuovendo la relazione ridondante \emph{Posizione} è necessario specificare i costi dei settori per un evento nell'operazione \ref{oploc1},
  affinché si possa sfruttare poi la relazione \emph{Costo} per derivare il luogo dal settore.
  Dunque, si dovrà sempre fare l'operazione \ref{oploc1} seguita dalla \ref{opsec1}.\\
  Questa non è una forte restrizione poiché nella pratica si
  inserisce sempre l'evento con i relativi prezzi dei biglietti nello stesso momento.
\end{quote}
\begin{tabularx}{\textwidth}{|X|X|X|X|}
\hline
  \multicolumn{4}{|l|}{\textbf{Operazione \ref{oploc1} + \ref{opsec1} (obbligatorio senza ridondanza)}} \\
\hline
\textbf{Concetto} & \textbf{Costrutto} & \textbf{Accessi} & \textbf{Tipo}\\
\hline
Evento & Entità & 1 & S \\
\hline
Costo & Relazione & 6 & S \\
\hline
\end{tabularx}
\newline
\vspace*{1em}
\newline
\begin{tabularx}{\textwidth}{|X|X|X|X|}
\hline
  \multicolumn{4}{|l|}{\textbf{Operazioni \ref{oploc2}, \ref{oploc4} e \ref{oploc5}}} \\
\hline
\textbf{Concetto} & \textbf{Costrutto} & \textbf{Accessi} & \textbf{Tipo}\\
\hline
Evento & Entità & 1 & L \\
\hline
Luogo & Entità & 1 & L \\
\hline
\end{tabularx}
\newline
\vspace*{1em}
\newline
\begin{tabularx}{\textwidth}{|X|X|X|X|}
\hline
  \multicolumn{4}{|l|}{\textbf{Operazione \ref{oploc3}}} \\
\hline
\textbf{Concetto} & \textbf{Costrutto} & \textbf{Accessi} & \textbf{Tipo}\\
\hline
Evento & Entità & 5 & L \\
\hline
Luogo & Entità & 2 & L \\
\hline
Spettacolo & Entità & 3 & L \\
\hline
\end{tabularx}

\subsubsection*{Tavole degli accessi rimuovendo la ridondanza}

\begin{tabularx}{\textwidth}{|X|X|X|X|}
\hline
  \multicolumn{4}{|l|}{\textbf{Operazione \ref{oploc1} + \ref{opsec1} (obbligatorio senza ridondanza)}} \\
\hline
\textbf{Concetto} & \textbf{Costrutto} & \textbf{Accessi} & \textbf{Tipo}\\
\hline
Evento & Entità & 1 & S \\
\hline
Costo & Relazione & 6 & S \\
\hline
\end{tabularx}
\newline
\vspace*{1em}
\newline
\begin{tabularx}{\textwidth}{|X|X|X|X|}
\hline
  \multicolumn{4}{|l|}{\textbf{Operazioni \ref{oploc2}, \ref{oploc4} e \ref{oploc5}}} \\
\hline
\textbf{Concetto} & \textbf{Costrutto} & \textbf{Accessi} & \textbf{Tipo}\\
\hline
Evento & Entità & 1 & L \\
\hline
Costo & Relazione & 6 & L \\
\hline
Settore & Entità & 6 & L \\
\hline
Luogo & Entità & 1 & L \\
\hline
\end{tabularx}
\newline
\vspace*{1em}
\newline
\begin{tabularx}{\textwidth}{|X|X|X|X|}
\hline
  \multicolumn{4}{|l|}{\textbf{Operazione \ref{oploc3}}} \\
\hline
\textbf{Concetto} & \textbf{Costrutto} & \textbf{Accessi} & \textbf{Tipo}\\
\hline
Evento & Entità & 5 & L \\
\hline
Costo & Relazione & 30 & L \\
\hline
Settore & Entità & 3 & L \\
\hline
Luogo & Entità & 2 & L \\
\hline
\end{tabularx}
\newline
\vspace*{1em}
\newline
Dunque, il costo \textbf{mensile} delle operazioni ammonta a (assumendo un
coefficiente moltiplicativo $2$ per le operazioni di scrittura):
\begin{center}
\begin{tabular}{ |c | r| }
  \hline
  \multicolumn{2}{|c|}{\textbf{Costo con ridondanza}} \\
  \hline
  \textbf{Operazione} & \multicolumn{1}{c|}{\textbf{Costo}} \\
  \hline
  Operazione \ref{oploc1} + \ref{opsec1} & $2 \cdot 7 \cdot 4 = 56$ \\
  \hline
  Operazione \ref{oploc2} & $2 \cdot 2 100 = 4 200$ \\
  \hline
  Operazione \ref{oploc3} & $10 \cdot 15 \cdot 31 = 4 650$ \\
  \hline
  Operazione \ref{oploc4} & $2 \cdot 8 = 16$ \\
  \hline
  Operazione \ref{oploc5} & $2 \cdot 5 = 10$ \\
  \hline
  \textbf{Totale} & $8 932$ \\
  \hline
\end{tabular}
\end{center}
\begin{center}
\begin{tabular}{ |c | r| }
  \hline
  \multicolumn{2}{|c|}{\textbf{Costo senza ridondanza}} \\
  \hline
  \textbf{Operazione} & \multicolumn{1}{c|}{\textbf{Costo}} \\
  \hline
  Operazione \ref{oploc1} + \ref{opsec1} & $2 \cdot 7 \cdot 4 = 56$ \\
  \hline
  Operazione \ref{oploc2} & $14 \cdot 2 100 = 29 400$ \\
  \hline
  Operazione \ref{oploc3} & $40 \cdot 15 \cdot 31 = 18 600$ \\
  \hline
  Operazione \ref{oploc4} & $14 \cdot 8 = 112$ \\
  \hline
  Operazione \ref{oploc5} & $14 \cdot 5 = 70$ \\
  \hline
  \textbf{Totale} & $48 238$ \\
  \hline
\end{tabular}
\end{center}

Poiché il costo sale di un fattore $\approx5.4$, riteniamo sia opportuno mantenere la ridondanza per migliorare le prestazioni.

\subsubsection*{Eliminazione delle gerarchie}

Riguardo l'entità Persona e Gruppo riteniamo che non debbano essere accorpate e quindi rimpiazziamo la generalizzazione con una relazione.
Abbiamo ritenuto inoltre di creare il seguente vincolo (RV10): un artista può essere una persona oppure un gruppo.
Di seguito, viene mostrato lo schema concettuale ristrutturato.

\includegraphics[width=\textwidth]{ERRistrutturato.png}

\subsection{Normalizzazione}

Osservando la ristrutturazione dello schema concettuale si nota che tutte le relazioni sono binarie, quindi sono già nella forma normale Boyce e Codd.\\
Non esistono dipendenze non banali tra gli attributi delle entità.

\subsection{Traduzione verso il modello relazionale}
\begin{tabularx}{\textwidth}{|X|>{\raggedright\arraybackslash}X|}
\hline
  \textbf{Entità - Relazione} & \textbf{Traduzione}\\
\hline
  Persona & Persona(\underline{id}, nome, cognome, data\_nascita) \\
\hline
  Gruppo & Gruppo(\underline{id}, data\_formazione) \\
\hline
  Persona Gruppo Appartenenza & Persona\_Gruppo\_Appartenenza (\underline{persona}, \underline{gruppo}) \\
\hline
  Artista & Artista(\underline{id}, \underline{nome\_arte}, persona, gruppo) \\
\hline
  Spettacolo & Spettacolo(\underline{id}, titolo, \underline{artista}, prezzo\_siae, cachet) \\
\hline
  Luogo & Luogo(\underline{id}, tipo, nome, indirizzo, citta, prezzo) \\
\hline
  Evento & Evento(\underline{id}, \underline{spettacolo}, \underline{luogo}, titolo, inizio, fine) \\
\hline
  Settore & Settore(\underline{id}, \underline{luogo}, nome, capienza) \\
\hline
  Posto & Posto(\underline{id}, \underline{settore}, fila, numero) \\
\hline
  Settore Evento Costo & Settore\_Evento\_Costo(\underline{settore}, \underline{evento}, prezzo) \\
\hline
  Fornitore & Fornitore(\underline{id}, nome, descrizione) \\
\hline
  Evento Fornitore Servizio & Evento\_Fornitore\_Servizio(\underline{fornitore}, tipo, \underline{evento}, prezzo) \\
\hline
  Biglietto & Biglietto(\underline{codice}, nominativo, \underline{posto}, \underline{evento}) \\
\hline
\end{tabularx}

\includegraphics[width=\textwidth]{schema.png}

\section{Codice SQL}

\subsection{Definizione dello schema}
\lstinputlisting{schema.sql}

\vspace*{1em}

\subsection{Definizione delle viste}
\lstinputlisting{views.sql}

\vspace*{1em}

\subsection{Codifica delle operazioni}
Nelle seguenti interrogazioni, i \texttt{...} si riferiscono a un input dato da un utente o da un'applicazione.
Due \texttt{...} diversi si riferiscono a due input diversi.

\subsubsection{Inserire un artista}
\lstinputlisting{queries/insert-artist.sql}

\subsubsection{Inserire uno spettacolo}
\lstinputlisting{queries/insert-show.sql}

\subsubsection{Inserire un fornitore di servizi}
\lstinputlisting{queries/insert-service-provider.sql}

\subsubsection{Inserire un luogo suddiviso in settori}

Per prima cosa inseriamo un luogo:
\lstinputlisting{queries/insert-venue.sql}
Successivamente i settori presenti:
\lstinputlisting{queries/insert-sector.sql}
Infine i posti all'interno di ogni settore:
\lstinputlisting{queries/insert-sector-seat.sql}

\subsubsection{Inserire un evento}
\lstinputlisting{queries/insert-event.sql}

\subsubsection{Assegnare un prezzo a ogni settore per un evento}
\lstinputlisting{queries/insert-sector-event-price.sql}

\subsubsection{Prenotare un servizio per un evento}
\lstinputlisting{queries/insert-service-provider-event.sql}

\subsubsection{Visualizzare un evento con il luogo dove si svolgerà}
\lstinputlisting{queries/search-event.sql}

\subsubsection{Vendere un biglietto}
\lstinputlisting{queries/insert-ticket.sql}

\subsubsection{Rimborsare un biglietto}

Poiché teniamo in memoria solo i biglietti acquistati, ``rimborsare'' un biglietto corrisponde a eliminare il biglietto dalla tabella \texttt{biglietto}.

\lstinputlisting{queries/delete-ticket.sql}

\subsubsection{Trovare un biglietto dato il posto e l'evento}
\lstinputlisting{queries/search-ticket-by-seat.sql}

\subsubsection{Visualizzare la capienza totale di un luogo}
\lstinputlisting{queries/search-venue-capacity.sql}

\subsubsection{Visualizzare i luoghi che soddisfano una capienza}
\lstinputlisting{queries/search-venue-capacity-limit.sql}

\subsubsection{Visualizzare tutti gli eventi di un artista}
Si vogliono selezionare i titoli e il nome del luogo di tutti gli eventi a cui
partecipa un dato artista.
\lstinputlisting{queries/search-events-by-artist.sql}

\subsubsection{Visualizzare tutti gli eventi di uno spettacolo}
\lstinputlisting{queries/search-events-by-show.sql}

\subsubsection{Visualizzare le spese di un evento}
Si vogliono mostrare tutte le spese dato un evento: i costi dei diritti SIAE e del
cachet dell'artista, il prezzo dell'affitto del luogo e tutti i costi relativi ai
servizi necessari.
\lstinputlisting{queries/search-expenses-by-event.sql}

\subsubsection{Visualizzare gli incassi di un evento}
\lstinputlisting{queries/search-incoming-by-event.sql}

\subsubsection{Visualizzare i guadagni di un evento}
Questa interrogazione non è altro che la differenza tra le spese e gli incassi di un evento.
\lstinputlisting{queries/search-earnings-by-event.sql}

\subsubsection{Visualizzare informazioni sui posti di un determinato evento}

\textbf{Visualizzare tutti i posti per un evento}\\
Si vogliono mostrare tutti i posti $(riga, colonna)$ per un dato evento.
\lstinputlisting{queries/search-seats-by-event.sql}
\textbf{Visualizzare tutti i posti prenotati per un evento}\\
Si vogliono mostrare tutti i posti $(riga, colonna)$ \textbf{prenotati} per un dato evento.
\lstinputlisting{queries/search-booked-seats-by-event.sql}
\textbf{Visualizzare tutti i posti liberi per un evento}\\
Si vogliono mostrare tutti i posti $(riga, colonna)$ \textbf{liberi} per un dato evento.
\lstinputlisting{queries/search-free-seats-by-event.sql}

In alternativa, per avere un miglioramento delle performance\footnote{
Vedi dati empirici su \href{https://stackoverflow.com/a/66785790}{StackOverflow}.
Nel nostro caso, abbiamo osservato che la query con il \texttt{NOT EXISTS}
impiega in media \texttt{5ms}, rispetto alla versione con il \texttt{NOT IN}
che può impiegare fino a \texttt{380ms}.}:
\lstinputlisting{queries/search-free-seats-by-event-ne.sql}

\textbf{Visualizzare la percentuale di biglietti venduti per evento}\\
Si vuole mostrare la percentuale di biglietti venduti per un singolo evento.
\lstinputlisting{queries/percentage-sold-tickets-by-event.sql}

\subsubsection{Visualizzare la percentuale media di biglietti venduti da un determinato artista}
Si vuole mostrare la percentuale media di tutti i biglietti venduti per ogni
evento per un dato artista.
\lstinputlisting{queries/mean-percentage-sold-tickets-by-artist.sql}

\subsubsection{Visualizzare quali eventi hanno fatto sold out}
Si vogliono mostrare il titolo degli eventi che hanno fatto sold out.
\lstinputlisting{queries/search-sold-out-events.sql}
 
\subsubsection{Visualizzare gli eventi presenti in un arco temporale}
Si vuole mostrare l'elenco di tutti gli eventi che si svolgono un range di date.
\lstinputlisting{queries/search-events-by-date.sql}

\subsubsection{Visualizzare tutti gli eventi che si svolgeranno in un determinato luogo}
Si vuole mostrare l'elenco di tutti gli eventi che si svolgono in un determinato luogo.
\lstinputlisting{queries/search-events-by-venue.sql}

\subsubsection{Visualizzare i guadagni dato uno spettacolo}
Si vuole mostrare il guadagno di un singolo spettacolo.
\lstinputlisting{queries/search-earnings-by-show.sql}

\subsubsection{Visualizzare lo show che ha guadagnato di più}
Si vuole mostrare il guadagno più alto raggiunto da uno show e il suo titolo.
\lstinputlisting{queries/show-max-earnings.sql}

\section{Risultati delle interrogazioni}
Abbiamo creato un'istanza di \emph{PostgreSQL} tramite \emph{Docker} e l'abbiamo popolato in automatico attraverso un programma scritto in \emph{Go} usando la libreria \emph{Faker} per creare dati casuali.
In questo modo possiamo provare le nostre interrogazioni su grandi moli di dati.
Di seguito, mostriamo alcune delle tabelle risultanti dalle interrogazioni.

% BEGIN
\subsubsection*{Visualizzare i luoghi che soddisfano una capienza}

Risultato d'esempio per l'esecuzione dell'operazione \ref{search-venue-capacity-limit}.

\begin{center}
\begin{tabular}{|l|l|l|}
\hline
\bfseries ID & \bfseries Nome & \bfseries Capienza 
\csvreader[head to column names]{csv/search-venue-capacity-limit.csv}{}
{\\\hline\id\ & \nome & \capienza} \\
\hline
\multicolumn{3}{|c|}{$\ldots$}  \\
\hline
\end{tabular}
\end{center}
% BEGIN

\end{document}
